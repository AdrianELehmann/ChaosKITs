\documentclass{article}

% Font encoding.
\usepackage[T1]{fontenc}
\usepackage[ngerman]{babel}
\usepackage[utf8]{inputenc}

% Display math.
\usepackage{amsmath}
\usepackage{mathtools}
\usepackage{amssymb}

% Nice enumerations without wasting space above and below.
\usepackage{enumitem}
\setlist{nosep}

% Headline and bottomline.
\usepackage{scrpage2}
\pagestyle{scrheadings}
\clearscrheadfoot
\ihead{Karlsruhe Institute of Technology}
\chead{ChaosKITs}
\ohead{\pagemark}

% Colors, used for syntax highlighting.
% To print this document, set all colors to black!
\usepackage{xcolor}
\definecolor{keyword}{rgb}{0, 0, 1}
\definecolor{string}{rgb}{1, 0, 0}
\definecolor{comment}{rgb}{0.2, 0.6, 0.2}
\definecolor{identifier}{rgb}{0, 0, 0}

% Source code listings.
\usepackage{pxfonts}
\usepackage{listings}
\lstset{
	language={C++},
	numbers=left,
	stepnumber=1,
	numbersep=6pt,
	numberstyle=\footnotesize,
	breaklines=true,
	breakautoindent=true,
	breakatwhitespace=false,
	postbreak=\space,
	tabsize=2,
	basicstyle=\ttfamily\footnotesize,
	showspaces=false,
	showstringspaces=false,
	extendedchars=true,
	keywordstyle=\color{keyword}\bfseries,
	stringstyle=\color{string}\bfseries,
	commentstyle=\color{comment}\bfseries,
  identifierstyle=\color{identifier},
	frame=trbl
}
% Listings doesn't support UTF8. This is just enough for German umlauts.
\lstset{literate=%
  {Ö}{{\"O}}1
  {Ä}{{\"A}}1
  {Ü}{{\"U}}1
  {ß}{{\ss}}1
  {ü}{{\"u}}1
  {ä}{{\"a}}1
  {ö}{{\"o}}1
  {~}{{\textasciitilde}}1
}

% Don't waste space at the page borders.
\usepackage[top=2cm, bottom=2cm, left=2cm, right=1cm]{geometry}

% Multicol layout for the table of contents.
\usepackage{multicol}
\usepackage{multirow}

% Automatically have table fill horizontal space.
\usepackage{tabularx}

% New enviroment for remarks.
\newtheorem{bem}{Bemerkung}

% New commands for math operators.
% Binomial coefficients.
\renewcommand{\binom}[2]{
  \biggl(
  \begin{matrix}
    #1 \\
    #2
  \end{matrix}
  \biggr)
}
% Euler numbers, first kind.
\newcommand{\eulerI}[2]{
  \biggl\langle
  \begin{matrix}
    #1 \\
    #2
  \end{matrix}
  \biggr\rangle
}
% Euler numbers, second kind.
\newcommand{\eulerII}[2]{
  \biggl\langle
  \negthinspace
  \biggl\langle
  \begin{matrix}
    #1 \\
    #2
  \end{matrix}
  \biggr\rangle
  \negthinspace
  \biggr\rangle
}
% Stirling numbers, first kind.
\newcommand{\stirlingI}[2]{
  \biggl[
  \begin{matrix}
    #1 \\
    #2
  \end{matrix}
  \biggr]
}
% Stirling numbers, second kind.
\newcommand{\stirlingII}[2]{
  \biggl\{
  \begin{matrix}
    #1 \\
    #2
  \end{matrix}
  \biggr\}
}

% Title and author information.
\title{Team Contest Reference}
\author{ChaosKITs \\ Karlsruhe Institute of Technology}
\begin{document}

% Titlepage with table of contents.
\maketitle
\setlength{\columnsep}{1cm}
\begin{multicols}{2}
	\tableofcontents
\end{multicols}
\newpage

\section{Datenstrukturen}

\subsection{Union-Find}
\lstinputlisting{datastructures/unionFind.cpp}

\subsection{Segmentbaum}
\lstinputlisting{datastructures/segmentTree.cpp}
Mit \lstinline{update()} können ganze Intervalle geändert werden.
Dazu: Offset in den inneren Knoten des Baums speichern.

\subsection{Fenwick Tree}
\lstinputlisting{datastructures/fenwickTree.cpp}


\subsection{Range Minimum Query}
\lstinputlisting{datastructures/RMQ.cpp}

\subsection{STL-Tree}
\lstinputlisting{datastructures/stlTree.cpp}

\subsection{STL-Rope}
\lstinputlisting{datastructures/stlRope.cpp}

\section{Graphen}

\subsection{Minimale Spannbäume}
Benutze Algorithmus von \textsc{Kruskal} oder Algorithmus von \textsc{Prim}.
\begin{description}
	\item[Schnitteigenschaft] Für jeden Schnitt $C$ im Graphen gilt: Gibt es eine Kante $e$, die echt leichter ist als alle anderen Schnittkanten, so gehört diese zu allen minimalen Spannbäumen. ($\Rightarrow$ Die leichteste Kante in einem Schnitt kann in einem minimalen Spannbaum verwendet werden.)
	\item[Kreiseigenschaft] Für jeden Kreis $K$ im Graphen gilt: Die schwerste Kante auf dem Kreis ist nicht Teil des minimalen Spannbaums.
\end{description}

\subsection{Kürzeste Wege}

\subsubsection{Algorithmus von \textsc{Dijkstra}}
Kürzeste Pfade in Graphen ohne negative Kanten.
\lstinputlisting{graph/dijkstra.cpp}

\subsubsection{\textsc{Bellmann-Ford}-Algorithmus}
Kürzestes Pfade in Graphen mit negativen Kanten. Erkennt negative Zyklen.
\lstinputlisting{graph/bellmannFord.cpp}

\subsubsection{\textsc{Floyd-Warshall}-Algorithmus}
Alle kürzesten Pfade im Graphen.
\lstinputlisting{graph/floydWarshall.cpp}
\textsc{Floyd-Warshall} findet auch negative Kreise. Es existiert genau dann ein negativer Kreis, wenn \lstinline{dist[i][i] < 0} ist.

\subsection{Strongly Connected Components (\textsc{Tarjans}-Algorithmus)}
\lstinputlisting{graph/scc.cpp}

\subsection{Artikulationspunkte und Brücken}
\lstinputlisting{graph/articulationPoints.cpp}

\subsection{Eulertouren}
\begin{itemize}
	\item Zyklus existiert, wenn jeder Knoten geraden Grad hat (ungerichtet), bzw. bei jedem Knoten Ein- und Ausgangsgrad übereinstimmen (gerichtet).
	\item Pfad existiert, wenn alle bis auf (maximal) zwei Knoten geraden Grad haben (ungerichtet), bzw. bei allen Knoten bis auf zwei Ein- und Ausgangsgrad übereinstimmen, wobei einer eine Ausgangskante mehr hat (Startknoten) und einer eine Eingangskante mehr hat (Endknoten).
	\item \textbf{Je nach Aufgabenstellung überprüfen, wie isolierte Punkte interpretiert werden sollen.}
	\item Der Code unten läuft in Linearzeit. Wenn das nicht notwenidg ist (oder bestimmte Sortierungen verlangt werden), gehts mit einem \lstinline{set} einfacher.
\end{itemize}
\begin{figure}[h]
\begin{lstlisting}
VISIT(v):
	forall e=(v,w) in E
	delete e from E
	VISIT(w)
	print e
\end{lstlisting}
\caption{Idee für Eulerzyklen}
\end{figure}
\lstinputlisting{graph/euler.cpp}

\subsection{Lowest Common Ancestor}
\lstinputlisting{graph/LCA.cpp}

\subsection{Max-Flow (\textsc{Edmonds-Karp}-Algorithmus)}
\lstinputlisting{graph/edmondsKarp.cpp}

\subsubsection{Maximum Edge Disjoint Paths}
Finde die maximale Anzahl Pfade von $s$ nach $t$, die keine Kante teilen.
\begin{enumerate}
	\item Setze $s$ als Quelle, $t$ als Senke und die Kapazität jeder Kante auf 1.
	\item Der maximale Fluss entspricht der unterschiedlichen Pfade ohne gemeinsame Kanten.
\end{enumerate}

\subsubsection{Maximum Independent Paths}
Finde die maximale Anzahl Pfade von $s$ nach $t$, die keinen Knoten teilen.
\begin{enumerate}
	\item Setze $s$ als Quelle, $t$ als Senke und die Kapazität jeder Kante \emph{und jedes Knotens} auf 1.
	\item Der maximale Fluss entspricht der unterschiedlichen Pfade ohne gemeinsame Knoten.
\end{enumerate}

\subsection{Maximal Cardinatlity Bipartite Mathcing}
\lstinputlisting{graph/maxCarBiMatch.cpp}

\subsection{TSP}
\lstinputlisting{graph/TSP.cpp}

\subsection{Bitonic TSP}
\lstinputlisting{graph/bitonicTSP.cpp}


\section{Geometrie}

\subsection{Closest Pair}
\lstinputlisting{geometry/closestPair.cpp}

\subsection{Geraden}
\lstinputlisting{geometry/lines.cpp}

\subsection{Konvexe Hülle}
\lstinputlisting{geometry/convexHull.cpp}

\subsection{Formeln - \lstinline{std::complex}}
\lstinputlisting{geometry/formulars.cpp}

\section{Mathe}

\subsection{ggT, kgV, erweiterter euklidischer Algorithmus}
\lstinputlisting{math/gcd-lcm.cpp}
\lstinputlisting{math/extendedEuclid.cpp}

\paragraph{Multiplikatives Inverses von $x$ in $\mathbb{Z}/n\mathbb{Z}$}
Sei $0 \leq x < n$. Definiere $d := \gcd(x, n)$.\newline
\textbf{Falls $d = 1$:}
\begin{itemize}[nosep]
	\item Erweiterter euklidischer Algorithmus liefert $\alpha$ und $\beta$ mit
	$\alpha x + \beta n = 1$.
	\item Nach Kongruenz gilt $\alpha x + \beta n \equiv \alpha x \equiv 1 \mod n$.
	\item $x^{-1} :\equiv \alpha \mod n$
	\end{itemize}
\textbf{Falls $d \neq 1$:} Es existiert kein $x^{-1}$.
\lstinputlisting{math/multInv.cpp}

\subsection{Mod-Exponent über $\mathbb{F}_p$}
\lstinputlisting{math/modExp.cpp}

\subsection{Chinesischer Restsatz}
\begin{itemize}
	\item Extrem anfällig gegen Overflows. Evtl. häufig 128-Bit Integer verwenden.
	\item Direkte Formel für zwei Kongruenzen $x \equiv a \mod n$, $x \equiv b \mod m$:
	\[
		x \equiv a - y * n * \frac{a - b}{d} \mod \frac{mn}{d}
		\qquad \text{mit} \qquad
		d := ggT(n, m) = yn + zm
	\]
	Formel kann auch für nicht teilerfremde Moduli verwendet werden.
	Sind die Moduli nicht teilerfremd, existiert genau dann eine Lösung,
	wenn $a_i \equiv a_j \mod \gcd(m_i, m_j)$.
	In diesem Fall sind keine Faktoren
	auf der linken Seite erlaubt.
\end{itemize}
\lstinputlisting{math/chineseRemainder.cpp}

\subsection{Primzahltest \& Faktorisierung}
\lstinputlisting{math/primes.cpp}

\subsection{Primzahlsieb von \textsc{Eratosthenes}}
\lstinputlisting{math/primeSieve.cpp}

\subsection{\textsc{Euler}sche $\varphi$-Funktion}
\begin{itemize}[nosep]
	\item Zählt die relativ primen Zahlen $\leq n$.

	\item Multiplikativ:
	$\gcd(a,b) = 1 \Longrightarrow \varphi(a) \cdot \varphi(b) = \varphi(ab)$

	\item $p$ prim, $k \in \mathbb{N}$:
	$~\varphi(p^k) = p^k - p^{k - 1}$

	\item $n = p_1^{a_1} \cdot \ldots \cdot p_k^{a_k}$:
	$~\varphi(n) = n \cdot \left(1 - \frac{1}{p_1}\right) \cdot \ldots \cdot \left(1 - \frac{1}{p_k}\right)$
	Evtl. ist es sinnvoll obgien Code zum Faktorisieren zu benutzen und dann diese Formel anzuwenden.

	\item \textbf{\textsc{Euler}'s Theorem:}
	Seien $a$ und $m$ teilerfremd. Dann:
	$a^{\varphi(m)} \equiv 1 \mod m$\newline
	Falls $m$ prim ist, liefert das den \textbf{kleinen Satz von \textsc{Fermat}}:
	$a^{m} \equiv a \mod m$
\end{itemize}
\lstinputlisting{math/phi.cpp}

\subsection{Primitivwurzeln}
\begin{itemize}[nosep]
	\item Primitivwurzel modulo $n$ existiert genau dann wenn:
	\begin{itemize}[nosep]
		\item $n$ ist $1$, $2$ oder $4$, oder
		\item $n$ ist Potenz einer ungeraden Primzahl, oder
		\item $n$ ist das Doppelte einer Potenz einer ungeraden Primzahl.
	\end{itemize}

	\item Sei $g$ Primitivwurzel modulo $n$.
	Dann gilt:\newline
	Das kleinste $k$, sodass $g^k \equiv 1 \mod n$, ist $k = \varphi(n)$.
\end{itemize}
\lstinputlisting{math/primitiveRoot.cpp}

\subsection{Diskreter Logarithmus}
\lstinputlisting{math/discreteLogarithm.cpp}

\subsection{Binomialkoeffizienten}
\lstinputlisting{math/binomial.cpp}

\subsection{LGS über $\mathbb{F}_p$}
\lstinputlisting{math/lgsFp.cpp}

\subsection{LGS über $\mathbb{R}$}
\lstinputlisting{math/gauss.cpp}

\subsection{Polynome \& FFT}
Multipliziert Polynome $A$ und $B$.
\begin{itemize}[nosep]
	\item $\deg(A * B) = \deg(A) + \deg(B)$
	\item Vektoren \lstinline{a} und \lstinline{b} müssen mindestens Größe
	$\deg(A * B) + 1$ haben.
	Größe muss eine Zweierpotenz sein.
	\item Für ganzzahlige Koeffizienten: \lstinline{(int)round(real(a[i]))}
\end{itemize}
\lstinputlisting{math/fft.cpp}

\subsection{3D-Kugeln}
\lstinputlisting{math/spheres.cpp}

\subsection{Longest Increasing Subsequence}
\lstinputlisting{math/longestIncreasingSubsequence.cpp}

\subsection{Kombinatorik}

\begin{tabular}{ll}
	\toprule
	\multicolumn{2}{c}{Berühmte Zahlen} \\
	\midrule 
	\textsc{Fibonacci}	&
	$f(0) = 0 \quad
	f(1) = 1 \quad
	f(n+2) = f(n+1) + f(n)$ \\

	\textsc{Catalan}	&
	$C_0 = 1 \qquad
	C_n = \sum\limits_{k = 0}^{n - 1} C_kC_{n - 1 - k} =
	\frac{1}{n + 1}\binom{2n}{n} = \frac{2(2n - 1)}{n+1} \cdot C_{n-1}$ \\

	\textsc{Euler} I &
	$\eulerI{n}{0} = \eulerI{n}{n-1} = 1 \qquad
	\eulerI{n}{k} = (k+1) \eulerI{n-1}{k} + (n-k) \eulerI{n-1}{k-1} $ \\

	\textsc{Euler} II &
	$\eulerII{n}{0} = 1 \quad
	\eulerII{n}{n} = 0 \quad
	\eulerII{n}{k} = (k+1) \eulerII{n-1}{k} + (2n-k-1) \eulerII{n-1}{k-1}$ \\

	\textsc{Stirling} I &
	$\stirlingI{0}{0} = 1 \qquad
	\stirlingI{n}{0} = \stirlingI{0}{n} = 0 \qquad
	\stirlingI{n}{k} = \stirlingI{n-1}{k-1} + (n-1) \stirlingI{n-1}{k}$ \\

	\textsc{Stirling} II &
	$\stirlingII{n}{1} = \stirlingII{n}{n} = 1 \qquad
	\stirlingII{n}{k} = k \stirlingII{n-1}{k} + \stirlingII{n-1}{k-1}$ \\

	\textsc{Bell} &
	$B_1 = 1 \qquad
	B_n = \sum\limits_{k = 0}^{n - 1} B_k\binom{n-1}{k}
	= \sum\limits_{k = 0}^{n}\stirlingII{n}{k}$\\

	\textsc{Int. Partitions} &
	$f(1,1) = 1 \quad
	f(n,k) = 0 \text{ für } k > n \quad
	f(n,k) = f(n-k,k) + f(n,k-1)$ \\
	\bottomrule
\end{tabular}

\paragraph{\textsc{Zeckendorfs} Theorem}
Jede positive natürliche Zahl kann eindeutig als Summe einer oder mehrerer
verschiedener \textsc{Fibonacci}-Zahlen geschrieben werden, sodass keine zwei
aufeinanderfolgenden \textsc{Fibonacci}-Zahlen in der Summe vorkommen.\\
\emph{Lösung:} Greedy, nimm immer die größte \textsc{Fibonacci}-Zahl, die noch
hineinpasst.

\paragraph{\textsc{Catalan}-Zahlen}
\begin{itemize}[nosep]
	\item Die erste und dritte angegebene Formel sind relativ sicher gegen Overflows.
	\item Die erste Formel kann auch zur Berechnung der \textsc{Catalan}-Zahlen
	bezüglich eines Moduls genutzt werden.
	\item Die \textsc{Catalan}-Zahlen geben an: $C_n =$
	\begin{itemize}[nosep]
		\item Anzahl der Binärbäume mit $n$ nicht unterscheidbaren Knoten.
		\item Anzahl der validen Klammerausdrücke mit $n$ Klammerpaaren.
		\item Anzahl der korrekten Klammerungen von $n+1$ Faktoren.
		\item Anzahl der Möglichkeiten ein konvexes Polygon mit $n + 2$ Ecken in
		Dreiecke zu zerlegen.
		\item Anzahl der monotonen Pfade (zwischen gegenüberliegenden Ecken) in
		einem $n \times n$-Gitter, die nicht die Diagonale kreuzen.
	\end{itemize}
\end{itemize}

\paragraph{\textsc{Euler}-Zahlen 1. Ordnung}
Die Anzahl der Permutationen von $\{1, \ldots, n\}$ mit genau $k$ Anstiegen.
Für die $n$-te Zahl gibt es $n$ mögliche Positionen zum Einfügen.
Dabei wird entweder ein Ansteig in zwei gesplitted oder ein Anstieg um $n$ ergänzt.

\paragraph{\textsc{Euler}-Zahlen 2. Ordnung}
Die Anzahl der Permutationen von $\{1,1, \ldots, n,n\}$ mit genau $k$ Anstiegen.

\paragraph{\textsc{Stirling}-Zahlen 1. Ordnung}
Die Anzahl der Permutationen von $\{1, \ldots, n\}$ mit genau $k$ Zyklen.
Es gibt zwei Möglichkeiten für die $n$-te Zahl. Entweder sie bildet einen eigene Zyklus, oder sie kann an jeder Position in jedem Zyklus einsortiert werden.

\paragraph{\textsc{Stirling}-Zahlen 2. Ordnung}
Die Anzahl der Möglichkeiten $n$ Elemente in $k$ nichtleere Teilmengen zu zerlegen.
Es gibt $k$ Möglichkeiten die $n$ in eine $n-1$-Partition einzuordnen.
Dazu kommt der Fall, dass die $n$ in ihrer eigenen Teilmenge (alleine) steht.

\paragraph{\textsc{Bell}-Zahlen}
Anzahl der Partitionen von $\{1, \ldots, n\}$.
Wie \textsc{Striling}-Zahlen 2. Ordnung ohne Limit durch $k$.

\paragraph{Integer Partitions}
Anzahl der Teilmengen von $\mathbb{N}$, die sich zu $n$ aufaddieren mit maximalem Elment $\leq k$.\\

\begin{tabular}{lcr}
	\toprule
	\multicolumn{3}{c}{Binomialkoeffizienten} \\
	\midrule
	$\binom{n}{k} = \frac{n!}{k!(n - k)!}$ &
	$\binom{n}{k} = \binom{n - 1}{k} + \binom{n - 1}{k - 1}$ &
	$\sum\limits_{k = 0}^n\binom{r + k}{k} = \binom{r + n + 1}{n}$ \\

	$\sum\limits_{k = 0}^n \binom{n}{k} = 2^n$ &
	$\binom{n}{m}\binom{m}{k} = \binom{n}{k}\binom{n - k}{m - k}$ &
	$\binom{n}{k} = (-1)^k \binom{k - n - 1}{k}$ \\

	$\binom{n}{k} = \binom{n}{n - k}$ &
	$\sum\limits_{k = 0}^n \binom{k}{m} = \binom{n + 1}{m + 1}$ &
	$\sum\limits_{i = 0}^n \binom{n}{i}^2 = \binom{2n}{n}$ \\

	$\binom{n}{k} = \frac{n}{k}\binom{n - 1}{k - 1}$ &
	$\sum\limits_{k = 0}^n \binom{r}{k}\binom{s}{n - k} = \binom{r + s}{n}$ &
	$\sum\limits_{i = 1}^n \binom{n}{i} F_i = F_{2n} \quad F_n = n\text{-th Fib.}$ \\
	\bottomrule
\end{tabular}

\begin{tabular}{c|cccc}
	\toprule
	\multicolumn{5}{c}{The Twelvefold Way (verteile $n$ Bälle auf $k$ Boxen)} \\
	\midrule
	Bälle & identisch & unterscheidbar & identisch      & unterscheidbar \\
	Boxen & identisch & identisch      & unterscheidbar & unterscheidbar \\
	\midrule
	- &
	$p_k(n)$ &
	$\sum\limits_{i = 0}^k \stirlingII{n}{i}$ &
	$\binom{n + k - 1}{k - 1}$ &
	$k^n$ \\

	size $\geq 1$ &
	$p(n, k)$ &
	$\stirlingII{n}{k}$ &
	$\binom{n - 1}{k - 1}$ &
	$k! \stirlingII{n}{k}$ \\

	size $\leq 1$ &
	$[n \leq k]$ &
	$[n \leq k]$ &
	$\binom{k}{n}$ &
	$n! \binom{k}{n}$ \\
	\midrule
	\multicolumn{5}{l}{
		$p_k(n)$: \#Anzahl der Partitionen von $n$ in $\leq k$ positive Summanden.
	} \\
	\multicolumn{5}{l}{
		$p(n, k)$: \#Anzahl der Partitionen von $n$ in genau $k$ positive Summanden.
	} \\
	\multicolumn{5}{l}{
		$[\text{Bedingung}]$: \lstinline{return Bedingung ? 1 : 0;}
	} \\
	\bottomrule
\end{tabular}
\vspace{5mm}

\begin{tabular}{ll}
	\toprule
	\multicolumn{2}{c}{Verschiedenes} \\
	\midrule
	Türme von Hanoi, minimale Schirttzahl: &
	$T_n = 2^n - 1$ \\

	\#Regionen zwischen $n$ Gearden	&
	$\frac{n\left(n + 1\right)}{2} + 1$ \\

	\#abgeschlossene Regionen zwischen $n$ Geraden &
	$\frac{n^2 - 3n + 2}{2}$ \\

	\#markierte, gewurzelte Bäume	&
	$n^{n-1}$ \\

	\#markierte, nicht gewurzelte Bäume	&
	$n^{n-2}$ \\

	\#Wälder mit $k$ gewurzelten Bäumen	&
	$\frac{k}{n}\binom{n}{k}n^{n-k}$ \\
	\bottomrule
\end{tabular}
\vspace{5mm}

\begin{tabular}{l|l|l}
	\toprule
	\multicolumn{3}{c}{Reihen} \\
	\midrule
	$\sum\limits_{i = 1}^n i = \frac{n(n+1)}{2}$ &
	$\sum\limits_{i = 1}^n i^2 = \frac{n(n + 1)(n + 2)}{6}$ & 
	$\sum\limits_{i = 1}^n i^3 = \frac{n^2 (n + 1)^2}{4}$ \\

	$\sum\limits_{i = 0}^n c^i = \frac{c^{n + 1} - 1}{c - 1} \quad c \neq 1$ &
	$\sum\limits_{i = 0}^\infty c^i = \frac{1}{1 - c} \quad \vert c \vert < 1$ &
	$\sum\limits_{i = 1}^\infty c^i = \frac{c}{1 - c} \quad \vert c \vert < 1$ \\

	\multicolumn{2}{l|}{
		$\sum\limits_{i = 0}^n ic^i = \frac{nc^{n + 2} - (n + 1)c^{n + 1} + c}{(c - 1)^2} \quad c \neq 1$
	} &
	$\sum\limits_{i = 0}^\infty ic^i = \frac{c}{(1 - c)^2} \quad \vert c \vert < 1$ \\

	$H_n = \sum\limits_{i = 1}^n \frac{1}{i}$ &
	\multicolumn{2}{l}{
		$\sum\limits_{i = 1}^n iH_i = \frac{n(n + 1)}{2}H_n - \frac{n(n - 1)}{4}$
	} \\

	$\sum\limits_{i = 1}^n H_i = (n + 1)H_n - n$ &
	\multicolumn{2}{l}{
		$\sum\limits_{i = 1}^n \binom{i}{m}H_i =
		\binom{n + 1}{m + 1} \left(H_{n + 1} - \frac{1}{m  + 1}\right)$
	} \\
	\bottomrule
\end{tabular}

\subsection{Satz von \textsc{Sprague-Grundy}}
Weise jedem Zustand $X$ wie folgt eine \textsc{Grundy}-Zahl $g\left(X\right)$ zu:
\[
	g\left(X\right) := \min\left\{
		\mathbb{Z}_0^+ \setminus
		\left\{g\left(Y\right) \mid Y \text{ von } X \text{ aus direkt erreichbar}\right\}
	\right\} 
\]
$X$ ist genau dann gewonnen, wenn $g\left(X\right) > 0$ ist.\\\\
Wenn man $k$ Spiele in den Zuständen $X_1, \ldots, X_k$ hat, dann ist die \textsc{Grundy}-Zahl des Gesamtzustandes $g\left(X_1\right) \oplus \ldots \oplus g\left(X_k\right)$.

\subsection{Big Integers}
\lstinputlisting{math/bigint.cpp}

\section{Strings}

\subsection{\textsc{Knuth-Morris-Pratt}-Algorithmus}
\lstinputlisting{string/kmp.cpp}

\subsection{\textsc{Aho-Corasick}-Automat}
\lstinputlisting{string/ahoCorasick.cpp}

\subsection{Trie}
\lstinputlisting{string/trie.cpp}

\subsection{Suffix-Baum}
\lstinputlisting{string/suffixTree.cpp}

\subsection{Suffix-Array}
\lstinputlisting{string/suffixArray.cpp}

\subsection{Suffix-Automaton}
\lstinputlisting{string/suffixAutomaton.cpp}
\begin{itemize}[nosep]
	\item \textbf{Ist \lstinline{w} Substring von \lstinline{s}?}
	Baue Automaten für \lstinline{s} und wende ihn auf \lstinline{w} an.
	Wenn alle Übergänge vorhanden sind, ist \lstinline{w} Substring von \lstinline{s}.

	\item \textbf{Ist \lstinline{w} Suffix von \lstinline{s}?}
	Wie oben.
	Überprüfe am Ende, ob aktueller Zustand ein Terminal ist.

	\item \textbf{Anzahl verschiedener Substrings.}
	Jeder Pfad im Automaten entspricht einem Substring.
	Für einen Knoten ist die Anzahl der ausgehenden Pfade gleich der Summe über die Anzahlen der Kindknoten plus 1.
	Der letzte Summand ist der Pfad, der in diesem Knoten endet.

	\item \textbf{Wie oft taucht \lstinline{w} in \lstinline{s} auf?}
	Sei \lstinline{p} der Zustand nach Abarbeitung von \lstinline{w}.
	Lösung ist Anzahl der Pfade, die in \lstinline{p} starten und in einem Terminal enden.
	Diese Zahl lässt sich wie oben rekursiv berechnen.
	Bei jedem Knoten darf nur dann plus 1 gerechnet werden, wenn es ein Terminal ist.
\end{itemize}

\subsection{Longest Common Subsequence}
\lstinputlisting{string/longestCommonSubsequence.cpp}

\subsection{Rolling Hash}
\lstinputlisting{string/rollingHash.cpp}

\section{Java}
\lstset{language=Java}

\subsection{Introduction}

\begin{itemize}
\item Compilen: \lstinline{javac main.java}
\item Ausführen: \lstinline{java main < sample.in}
\item Eingabe:
\lstinline{Scanner} ist sehr langsam. Bei großen Eingaben muss ein Buffered Reader verwendet werden.
\begin{lstlisting}
Scanner in = new Scanner(System.in); // java.util.Scanner
String line = in.nextLine(); // Liest die nächste Zeile.
int num = in.nextInt(); // Liest das nächste Token als int.
double num2 = in.nextDouble(); // Liest das nächste Token als double.
\end{lstlisting}
\item Ausgabe:
\begin{lstlisting}
// Ausgabe in StringBuilder schreiben und am Ende alles auf einmal ausgeben. -> Viel schneller.
StringBuilder sb = new StringBuilder(); // java.lang.StringBuilder
sb.append("Hallo Welt");
System.out.print(sb.toString());
\end{lstlisting}
\end{itemize}

\subsection{BigInteger}
\begin{lstlisting}
// Berechnet this +,*,/,- val.
BigInteger add(BigInteger val), multiply(BigInteger val), divide(BigInteger val), substract(BigInteger val)

// Berechnet this^e.
BigInteger pow(BigInteger e)

// Bit-Operationen.
BigInteger and(BigInteger val), or(BigInteger val), xor(BigInteger val), not(), shiftLeft(int n), shiftRight(int n)

// Berechnet den ggT von abs(this) und abs(val).
BigInteger gcd(BigInteger val)

// Berechnet this mod m, this^-1 mod m, this^e mod m.
BigInteger mod(BigInteger m), modInverse(BigInteger m), modPow(BigInteger e, BigInteger m)

// Berechnet die nächste Zahl, die größer und wahrscheinlich prim ist.
BigInteger  nextProbablePrime()

// Berechnet int/long/float/double-Wert. Ist die Zahl zu großen werden die niedrigsten Bits konvertiert.
int intValue(), long longValue(), float floatValue(), double doubleValue() 
\end{lstlisting}
\lstset{language=C++}
\section{Sonstiges}

\subsection{2-SAT}
\begin{enumerate}
	\item Bedingungen in 2-CNF formulieren.
	\item Implikationsgraph bauen, $\left(a \vee b\right)$ wird zu $\neg a \Rightarrow b$ und $\neg b \Rightarrow a$.
	\item Finde die starken Zusammenhangskomponenten.
	\item Genau dann lösbar, wenn keine Variable mit ihrer Negation in einer SCC liegt.
\end{enumerate}

\subsection{Sortieren in Linearzeit}
Wenn die Eingabe aus einem kleinen Intervall $\left[0, n\right)$ stammt ist Bucketsort schneller.

\subsubsection{Bucketsort}
\lstinputlisting{sonstiges/bucketSort.cpp}

\subsubsection{LSD-Radixsort}
\lstinputlisting{sonstiges/radixsort.cpp}

\subsection{Bit Operations}
\lstinputlisting{sonstiges/bitOps.cpp}

\subsection{Roman-Literal-Converting}
\lstinputlisting{sonstiges/Roman.cpp}

\subsection{Josephus-Problem}
$n$ Personen im Kreis, jeder $k$-te wird erschossen.
\begin{description}
	\item[Spezialfall $k=2$:] Betrachte Binärdarstellung von $n$. Für $n = 1b_1b_2b_3..b_n$ ist $b_1b_2b_3..b_n1$ die Position des letzten Überlebenden. (Rotiere $n$ um eine Stelle nach links)
	\item[Allgemein:] Sei $F(n,k)$ die Position des letzten Überlebenden. Nummeriere die Personen mit $0, 1, \ldots, n-1$. Nach Erschießen der $k$-ten Person, hat der Kreis noch Größe $n-1$ und die Position des Überlebenden ist jetzt $F(n-1,k)$. Also: $F(n,k) = (F(n-1,k)+k)\%n$. Basisfall: $F(1,k) = 0$. 
\end{description}
\textbf{Beachte bei der Ausgabe, dass die Personen im ersten Fall von $1, \ldots, n$ nummeriert sind, im zweiten Fall von $0, \ldots, n-1$!}

\section{Convenience-Methoden}

\subsection{Zeileneingabe}
\lstinputlisting{convenience/split.cpp}

\subsection{Sonstiges}
\begin{lstlisting}
// Alles-Header.
#include <bits/stdc++.h>

// Setzt das deutsche Tastaturlayout.
setxkbmap de

// Schnelle Ein-/Ausgabe mit cin/cout.
ios::sync_with_stdio(false);

// Set mit eigener Sortierfunktion. Typ muss nicht explizit angegeben werden.
set<point2, decltype(comp)> set1(comp);

// PI
#define PI (2*acos(0))

// STL-Debugging, Compiler flags.
-D_GLIBCXX_DEBUG
#define _GLIBCXX_DEBUG
\end{lstlisting}


\end{document}