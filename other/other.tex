\section{Sonstiges}

\subsection{2-SAT}
\begin{enumerate}
	\item Bedingungen in 2-CNF formulieren.
	\item Implikationsgraph bauen, $\left(a \vee b\right)$ wird zu $\neg a \Rightarrow b$ und $\neg b \Rightarrow a$.
	\item Finde die starken Zusammenhangskomponenten.
	\item Genau dann lösbar, wenn keine Variable mit ihrer Negation in einer SCC liegt.
\end{enumerate}

\subsection{Zeileneingabe}
\lstinputlisting{other/split.cpp}

\subsection{Bit Operations}
\lstinputlisting{other/bitOps.cpp}

\subsection{Josephus-Problem}
$n$ Personen im Kreis, jeder $k$-te wird erschossen.
\begin{description}
	\item[Spezialfall $k=2$:] Betrachte Binärdarstellung von $n$.
	Für $n = 1b_1b_2b_3..b_n$ ist $b_1b_2b_3..b_n1$ die Position des letzten Überlebenden.
	(Rotiere $n$ um eine Stelle nach links)
	\lstinputlisting{other/josephus2.cpp}
	\item[Allgemein:] Sei $F(n,k)$ die Position des letzten Überlebenden.
	Nummeriere die Personen mit $0, 1, \ldots, n-1$.
	Nach Erschießen der $k$-ten Person, hat der Kreis noch Größe $n-1$ und die Position des Überlebenden ist jetzt $F(n-1,k)$.
	Also: $F(n,k) = (F(n-1,k)+k)\%n$. Basisfall: $F(1,k) = 0$. 
	\lstinputlisting{other/josephusK.cpp}
\end{description}
\textbf{Beachte bei der Ausgabe, dass die Personen im ersten Fall von $1, \ldots, n$ nummeriert sind, im zweiten Fall von $0, \ldots, n-1$!}

\subsection{Gemischtes}
\begin{itemize}
	\item \emph{\textsc{Johnsons} Reweighting Algorithmus:}
	Füge neue Quelle \lstinline{S} hinzu, mit Kanten mit Gewicht 0 zu allen Knoten.
	Nutze \textsc{Bellmann-Ford} zum Betsimmen der Entfernungen \lstinline{d[i]} von \lstinline{S} zu allen anderen Knoten.
	Stoppe, wenn es negative Zyklen gibt.
	Sonst ändere die gewichte von allen Kanten \lstinline{(u,v)} im ursprünglichen Graphen zu \lstinline{d[u]+w[u,v]-d[v]}.
	Dann sind alle Kantengewichte nichtnegativ, \textsc{Dijkstra} kann angewendet werden.
	\item Für ein System von Differenzbeschränkungen:
	Ändere alle Bedingungen in die Form $a-b \leq c$.
	Für jede Bedingung füge eine Kante \lstinline{(b,a)} mit Gweicht \lstinline{c} ein.
	Füge Quelle \lstinline{s} hinzu, mit Kanten zu allen Knoten mit Gewicht 0.
	Nutze \textsc{Bellmann-Ford}, um die kürzesten Pfade von \lstinline{s} aus zu finden.
	\lstinline{d[v]} ist mögliche Lösung für \lstinline{v}.
	\item Min-Weight-Vertex-Cover im bipartiten Graph:
	Partitioniere in \lstinline{A, B} und füge Kanten \lstinline{s -> A} mit Gewicht \lstinline{w(A)} und Kanten  \lstinline{B -> t} mit Gewicht \lstinline{w(B)} hinzu.
	Füge Kanten mit Kapazität $\infty$ von \lstinline{A} nach \lstinline{B} hinzu, wo im originalen Graphen Kanten waren.
	Max-Flow ist die Lösung.\newline
	Im Residualgraphen:
	\begin{itemize}
		\item Das Vertex-Cover sind die Knoten inzident zu den Brücken. \emph{oder}
		\item Die Knoten in \lstinline{A}, die \emph{nicht} von \lstinline{s} erreichber sind und die Knoten in \lstinline{B}, die von \lstinline{erreichber} sind.
	\end{itemize}
	\item Allgemeiner Graph:
	Das Komplement eines Vertex-Cover ist ein Independent Set.
	$\Rightarrow$ Max Weight Independent Set ist Komplement von Min Weight Vertex Cover.
	\item Bipartiter Graph:
	Min Vertex Cover (kleinste Menge Kanten, die alle Knoten berühren) = Max Matching.
	\item Bipartites Matching mit Gewichten auf linken Knoten.
	Minimiere Matchinggewicht.
	Lösung: Sortiere Knoten links aufsteigend nach Gewicht, danach nutze normlen Algorithmus (\textsc{Kuhn}, Seite \pageref{kuhn})
	\item \textbf{Tobi, cool down!}
\end{itemize}

\subsection{Sonstiges}
\begin{lstlisting}
// Alles-Header.
#include <bits/stdc++.h>

// Setzt das deutsche Tastaturlayout.
setxkbmap de

// Schnelle Ein-/Ausgabe mit cin/cout.
ios::sync_with_stdio(false);

// Set mit eigener Sortierfunktion. Typ muss nicht explizit angegeben werden.
set<point2, decltype(comp)> set1(comp);

// PI
#define PI (2*acos(0))

// STL-Debugging, Compiler flags.
-D_GLIBCXX_DEBUG
#define _GLIBCXX_DEBUG

// 128-Bit Integer. Muss zum Einlesen/Ausgeben in einen int oder long long gecastet werden.
__int128
\end{lstlisting}
