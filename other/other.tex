\section{Sonstiges}

\subsection{Zeileneingabe}
\lstinputlisting{other/split.cpp}

\subsection{Bit Operations}
\lstinputlisting{other/bitOps.cpp}

\subsection{Sonstiges}
\begin{lstlisting}
// Alles-Header.
#include <bits/stdc++.h>

// Setzt das deutsche Tastaturlayout.
setxkbmap de

// Schnelle Ein-/Ausgabe mit cin/cout.
ios::sync_with_stdio(false);

// Set mit eigener Sortierfunktion. Typ muss nicht explizit angegeben werden.
set<point2, decltype(comp)> set1(comp);

// PI
#define PI (2*acos(0))

// STL-Debugging, Compiler flags.
-D_GLIBCXX_DEBUG
#define _GLIBCXX_DEBUG

// 128-Bit Integer/Float. Muss zum Einlesen/Ausgeben in einen int oder long long gecastet werden.
__int128, __float128
\end{lstlisting}

\subsection{Josephus-Problem}
$n$ Personen im Kreis, jeder $k$-te wird erschossen.
\begin{description}
	\item[Spezialfall $k=2$:] Betrachte Binärdarstellung von $n$.
	Für $n = 1b_1b_2b_3..b_n$ ist $b_1b_2b_3..b_n1$ die Position des letzten Überlebenden.
	(Rotiere $n$ um eine Stelle nach links)
	\lstinputlisting{other/josephus2.cpp}
	\item[Allgemein:] Sei $F(n,k)$ die Position des letzten Überlebenden.
	Nummeriere die Personen mit $0, 1, \ldots, n-1$.
	Nach Erschießen der $k$-ten Person, hat der Kreis noch Größe $n-1$ und die Position des Überlebenden ist jetzt $F(n-1,k)$.
	Also: $F(n,k) = (F(n-1,k)+k)\%n$. Basisfall: $F(1,k) = 0$. 
	\lstinputlisting{other/josephusK.cpp}
\end{description}
\textbf{Beachte bei der Ausgabe, dass die Personen im ersten Fall von $1, \ldots, n$ nummeriert sind, im zweiten Fall von $0, \ldots, n-1$!}

\subsection{Gemischtes}
\begin{itemize}
	\item \textbf{\textsc{Johnsons} Reweighting Algorithmus:}
	Füge neue Quelle \lstinline{S} hinzu, mit Kanten mit Gewicht 0 zu allen Knoten.
	Nutze \textsc{Bellmann-Ford} zum Betsimmen der Entfernungen \lstinline{d[i]} von \lstinline{S} zu allen anderen Knoten.
	Stoppe, wenn es negative Zyklen gibt.
	Sonst ändere die gewichte von allen Kanten \lstinline{(u,v)} im ursprünglichen Graphen zu \lstinline{d[u]+w[u,v]-d[v]}.
	Dann sind alle Kantengewichte nichtnegativ, \textsc{Dijkstra} kann angewendet werden.

	\item \textbf{System von Differenzbeschränkungen:}
	Ändere alle Bedingungen in die Form $a-b \leq c$.
	Für jede Bedingung füge eine Kante \lstinline{(b,a)} mit Gweicht \lstinline{c} ein.
	Füge Quelle \lstinline{s} hinzu, mit Kanten zu allen Knoten mit Gewicht 0.
	Nutze \textsc{Bellmann-Ford}, um die kürzesten Pfade von \lstinline{s} aus zu finden.
	\lstinline{d[v]} ist mögliche Lösung für \lstinline{v}.

	\item \textbf{Min-Weight-Vertex-Cover im bipartiten Graph:}
	Partitioniere in \lstinline{A, B} und füge Kanten \lstinline{s -> A} mit Gewicht \lstinline{w(A)} und Kanten  \lstinline{B -> t} mit Gewicht \lstinline{w(B)} hinzu.
	Füge Kanten mit Kapazität $\infty$ von \lstinline{A} nach \lstinline{B} hinzu, wo im originalen Graphen Kanten waren.
	Max-Flow ist die Lösung.\newline
	Im Residualgraphen:
	\begin{itemize}[nosep]
		\item Das Vertex-Cover sind die Knoten inzident zu den Brücken. \emph{oder}
		\item Die Knoten in \lstinline{A}, die \emph{nicht} von \lstinline{s} erreichber sind und die Knoten in \lstinline{B}, die von \lstinline{erreichber} sind.
	\end{itemize}

	\item \textbf{Allgemeiner Graph:}
	Das Komplement eines Vertex-Cover ist ein Independent Set.
	$\Rightarrow$ Max Weight Independent Set ist Komplement von Min Weight Vertex Cover.

	\item \textbf{Bipartiter Graph:}
	Min Vertex Cover (kleinste Menge Kanten, die alle Knoten berühren) = Max Matching.

	\item \textbf{Bipartites Matching mit Gewichten auf linken Knoten:}
	Minimiere Matchinggewicht.
	Lösung: Sortiere Knoten links aufsteigend nach Gewicht, danach nutze normlen Algorithmus (\textsc{Kuhn}, Seite \pageref{kuhn})

	\item \textbf{Satz von \textsc{Pick}:}
	Sei $A$ der Flächeninhalt eines einfachen Gitterpolygons, $I$ die Anzahl der Gitterpunkte im Inneren und $R$ die Anzahl der Gitterpunkte auf dem Rand.
	Es gilt:
	\[
		A = I + \frac{R}{2} - 1
	\]

	\item \textbf{Lemma von \textsc{Burnside}:}
	Sei $G$ eine endliche Gruppe, die auf der Menge $X$ operiert.
	Für jedes $g \in G$ sei $X^g$ die Menge der Fixpunkte bei Operation durch $g$, also $X^g = \{x \in X \mid g \bullet x = x\}$.
	Dann gilt für die Anzahl der Bahnen $[X/G]$ der Operation:
	\[
		[X/G] = \frac{1}{\vert G \vert} \sum_{g \in G} \vert X^g \vert
	\]
\end{itemize}

\subsection{Tipps \& Tricks}

\begin{itemize}
	\item Run Tim Error:
	\begin{itemize}
		\item Stack Overflow? Evtl. rekurisve Tiefensuche auf langem Pfad?
		\item Array-Grenzen überprüfen. Indizierung bei $0$ oder bei $1$ beginnen?
		\item Abbruchbedingung bei Rekursion?
		\item Evtl. Memory Limit Exceeded?
	\end{itemize}

	\item Gleitkommazahlen:
	\begin{itemize}
		\item \lstinline{NaN}? Evtl. ungültige Werte für mathematische Funktionen, z.B. \lstinline{acos(1.00000000000001)}?
		\item Flasches Runden bei negativen Zahlen? Abschneiden $\neq$ Abrunden!
		\item Output in wissenschaftlicher Notation (\lstinline{1e-25})?
		\item Kann \lstinline{-0.000} ausgegeben werden?
	\end{itemize}

	\item Wrong Answer:
	\begin{itemize}
		\item Lies Aufgabe erneut. Sorgfältig!
		\item Mehrere Testfälle in einer Datei? Probiere gleichen Testcase mehrfach hintereinander.
		\item Integer Overflow? Teste maximale Eingabegrößen und mache Überschlagsrechnung.
		\item Einabegrößen überprüfen. Sonderfälle ausprobieren.
		\begin{itemize}
			\item $n = 0$, $n = -1$, $n = 1$, $n = 2^{31}-1$, $n = -2^{31}$
			\item $n$ gerade/ungerade
			\item Graph ist leer/enthält nur einen Knoten.
			\item Liste ist leer/enthält nur ein Element.
			\item Graph ist Multigraph (enthält Schleifen/Mehrfachkanten).
			\item Sind Kanten gerichtet/ungerichtet?
			\item Polygon ist konkav/selbstschneidend.
		\end{itemize}
		\item Bei DP/Rekursion: Stimmt Basisfall?
		\item Unsicher bei benutzten STL-Funktionen?
	\end{itemize}
\end{itemize}
