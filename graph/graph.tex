\section{Graphen}

\subsection{Lowest Common Ancestor}
\lstinputlisting{graph/LCA.cpp}

\subsection{Kürzeste Wege}

\subsubsection{Algorithmus von \textsc{Dijkstra}}
Kürzeste Pfade in Graphen ohne negative Kanten.
\lstinputlisting{graph/dijkstra.cpp}

\subsubsection{\textsc{Bellmann-Ford}-Algorithmus}
Kürzestes Pfade in Graphen mit negativen Kanten. Erkennt negative Zyklen.
\lstinputlisting{graph/bellmannFord.cpp}

\subsubsection{\textsc{Floyd-Warshall}-Algorithmus}
Alle kürzesten Pfade im Graphen.
\lstinputlisting{graph/floydWarshall.cpp}

\subsection{Strongly Connected Components (\textsc{Tarjans}-Algorithmus)}
\lstinputlisting{graph/scc.cpp}

\subsection{Artikulationspunkte und Brücken}
\lstinputlisting{graph/articulationPoints.cpp}

\subsection{Eulertouren}
\begin{itemize}
	\item Zyklus existiert, wenn jeder Knoten geraden Grad hat (ungerichtet), bzw. bei jedem Knoten Ein- und Ausgangsgrad übereinstimmen (gerichtet).
	\item Pfad existiert, wenn alle bis auf (maximal) zwei Knoten geraden Grad haben (ungerichtet), bzw. bei allen Knoten bis auf zwei Ein- und Ausgangsgrad übereinstimmen, wobei einer eine Ausgangskante mehr hat (Startknoten) und einer eine Eingangskante mehr hat (Endknoten).
	\item \textbf{Je nach Aufgabenstellung überprüfen, wie isolierte Punkte interpretiert werden sollen.}
	\item Der Code unten läuft in Linearzeit. Wenn das nicht notwenidg ist (oder bestimmte Sortierungen verlangt werden), gehts mit einem \lstinline{set} einfacher.
\end{itemize}
\lstinputlisting{graph/euler.cpp}

\subsection{Max-Flow (\textsc{Edmonds-Karp}-Algorithmus)}
\lstinputlisting{graph/edmondsKarp.cpp}
